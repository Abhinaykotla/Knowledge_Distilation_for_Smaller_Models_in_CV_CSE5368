\section{Experiments}

\subsection{Training Details}

Several student models were configured based on the teacher model described in Section~\ref{sec:method:model_architecture},
applying the reduction and quantization techniques
introduced in Sections~\ref{sec:method:weight_reduction} and~\ref{sec:method:precision_quantization}.
All training was conducted on a single NVIDIA GeForce RTX 4070 Laptop GPU.
The implementation is based on the PyTorch framework \cite{paszke2019pytorch},
leveraging its extensive libraries and tools for deep learning.
The Adam optimizer \cite{kingma2014adam} was employed with a learning rate of 0.001.
A mini-batch size of 224 was used, and KL Divergence was utilized as the loss function,
effectively measuring the difference
between the distributions predicted by the teacher and those output by the student models.
Early stopping was applied with a patience of 3 epochs to prevent overfitting.

\subsection{Evaluation}

The performance of the teacher model is summarized in Table~\ref{tab:teacher_model}.
Table~\ref{tab:weight_reduction} shows the performance of the student models
along with the reduction in the number of parameters.
Additionally, Table~\ref{tab:precision_quantization} presents the performance
achieved through quantization of model weights and floating-point operations.

\begin{table}[ht]
\centering
\caption{Teacher Model Performance}
\label{tab:teacher_model}
\begin{tabular}{c|c}
    \noalign{\hrule height 1pt}
                            & \textit{Teacher model} \\ \hline
    Number of parameters    & 26,534,358 \\ 
    Accuracy                 & 85.77\% \\ 
    Training time per epoch (seconds) & 72.32 \\ 
    \noalign{\hrule height 1pt}
\end{tabular}
\end{table}

\begin{table}[ht]
\centering
\caption{Weight Reduction Comparison}
\label{tab:weight_reduction}
\begin{tabular}{c|c}
    \noalign{\hrule height 1pt}
                            & \textit{10 blocks, (512, 256, 64, 16)} \\ \hline
    Number of parameters    & 6,601,686 \\ 
    Accuracy                 & 83.23\% \\ 
    Training time per epoch (seconds) & 68.6 \\ 
    \noalign{\hrule height 1pt}
\end{tabular}
\\[10pt]
\begin{tabular}{c|c}
    \noalign{\hrule height 1pt}
                            & \textit{8 blocks, (256, 64, 16)} \\ \hline
    Number of parameters    & 1,648,470 \\ 
    Accuracy                 & \textbf{86.43\%} \\ 
    Training time per epoch (seconds) & 67.0 \\ 
    \noalign{\hrule height 1pt}
\end{tabular}
\\[10pt]
\begin{tabular}{c|c}
    \noalign{\hrule height 1pt}
                            & \textit{6 blocks, (128, 64, 16)} \\ \hline
    Number of parameters    & 408,854 \\ 
    Accuracy                 & 85.77\% \\ 
    Training time per epoch (seconds) & 66.7 \\ 
    \noalign{\hrule height 1pt}
\end{tabular}
\\[10pt]
\begin{tabular}{c|c}
    \noalign{\hrule height 1pt}
                            & \textit{4 blocks, (64, 32)} \\ \hline
    Number of parameters    & 98,294 \\ 
    Accuracy                 & 84.57\% \\ 
    Training time per epoch (seconds) & \textbf{64.7} \\ 
    \noalign{\hrule height 1pt}
\end{tabular}
\end{table}

\begin{table}[ht]
\centering
\caption{Precision Quantization on Teacher Model}
\label{tab:precision_quantization}
\begin{tabular}{c|cc}
    \noalign{\hrule height 1pt}
                        & \textit{fp16}        & \textit{fp8} \\ \hline
    Accuracy            & 85.54\%              & \textbf{86.38\%} \\ 
    Training time per epoch (seconds) & \textbf{61.3}    & 70.5 \\ 
    \noalign{\hrule height 1pt}
\end{tabular}
\end{table}

As expected, the training time per epoch decreases as the number of parameters is reduced.
The smallest student model achieves the fastest training time per epoch, at 64.7 seconds.
Interestingly, the performance does not strictly correlate with the number of parameters:
the 8-block student model outperforms others with an accuracy of 86.43\%,
highlighting that architectural choices and dataset characteristics
can outweigh mere parameter counts.

\begin{table}[ht]
\centering
\caption{Precision Quantization on 8-Block Model}
\label{tab:quantization_8_blocks}
\begin{tabular}{c|cc}
    \noalign{\hrule height 1pt}
                        & \textit{fp16}        & \textit{fp8} \\ \hline
    Accuracy            & \textbf{87.00\%}     & 86.07\% \\ 
    Training time per epoch (seconds) & \textbf{60.8}    & 69.0 \\ 
    \noalign{\hrule height 1pt}
\end{tabular}
\end{table}

Using the best-performing student model (8 blocks),
we further experimented with precision quantization.
The results in Table~\ref{tab:quantization_8_blocks} demonstrate that
the 8-block student model benefits significantly from fp16 precision.
It achieved an accuracy of 87.00\% with the fastest training time per epoch at 60.8 seconds,
while fp8 precision resulted in slightly lower accuracy (86.07\%) and a slower training time (69.0 seconds).
This highlights that selecting the appropriate precision can significantly improve
both model performance and computational efficiency.
