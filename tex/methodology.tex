\section{Methodology}

\subsection{Intel Image Dataset}

The Intel Image Classification Dataset \cite{intel_image_classification_kaggle} was curated
as part of a data hackathon organized
by Intel \footnote{https://www.intel.com/content/www/us/en/homepage.html}
on an online platform Kaggle \footnote{https://www.kaggle.com/},
with the aim of fostering engagement and innovation within data science communities.
The dataset comprises 25,000 RGB color images,
each with a resolution of 150 $\times$ 150 pixels,
depicting a variety of natural and urban scenes from around the world.
The images, originally captured by Jan Böttinger
on Unsplash \footnote{https://unsplash.com/},
are labeled into six distinct classes: buildings, forest, glacier, mountain, sea, and street.

The dataset is partitioned into three subsets: approximately 14,000 images for training,
3,000 images for testing, and 7,000 images for prediction tasks.
Each subset is provided as a separate zipped file through Kaggle.

The primary object of this dataset is to support participants
in developing and evaluating image classification models.
By providing a diverse and well-labeled collection of scenes,
the dataset encourages the advancement of robust classification algorithms
capable to distinguish objects in various environments with reasonable accuracy.

\subsection{Model Architecture}
\label{sec:method:model_architecture}

In this section, the architecture of the teacher model is described.
We implemented a custom \gls*{cnn} model, including the residual block \cite{he2016deep},
which hierarchically extracts and aggregates spatial features from the input images,
before mapping them to one of six scene categories.

The model consists of 12 residual blocks,
progressively increasing the channel capacity while reducing spatial resolution,
and a fully connected layer at the end as a classifier
which regularly decreases the number from 1024, 512, 256, 64, and 6.
In total, 26,534,358 parameters are in the model.

All operations in the model are performed using floating point 32,
which is the default precision in PyTorch \cite{paszke2019pytorch}
in most GPUs.

\subsection{Weight Reduction}
\label{sec:method:weight_reduction}

From the teacher model,
the number of residual blocks and the layers of the classifier are reduced.
In the perspective of weight reduction,
four reduction models are created, excluding the teacher model,
which are 10, 8, 6, and 4 residual blocks
and (512, 256, 64, 16), (256, 64, 16), (128, 64, 16), and (64, 32) layers of the classifiers.
The number of parameters in the models is 6,601,686, 1,648,470, 408,854, and 98,294, respectively.

\subsection{Precision Quantization}
\label{sec:method:precision_quantization}

The quantization is utilized in both model weights and floating point operations.
Two different levels of precision, floating point 16 and 8, are used.
To implement the quantization,
\texttt{torch.set\_default\_dtype()} and the \texttt{bnb}~\cite{bitsandbytes} Python framework
are used.