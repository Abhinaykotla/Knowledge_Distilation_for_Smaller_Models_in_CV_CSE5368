\section{Related Works}

In this section, we review performance optimization techniques
for the image classification task through three main approaches:
\textit{\ref{sec:rw:lightweight_backbones} lightweight backbones},
\textit{\ref{sec:rw:knowledge_distillation} knowledge distillation},
and \textit{\ref{sec:rw:quantization} quantization}.
The first approach focuses on the design of efficient architectures,
the second emphasizes the transfer of knowledge from a larger model to a smaller one,
and the third discusses the quantization of models,
a widely used technique for reducing model size
and improving performance on low-compute devices.

\subsection{Lightweight Backbones}
\label{sec:rw:lightweight_backbones}

Early breakthroughs in large-scale image classification were driven by deeper and wider \gls*{cnn}
architectures such as AlexNet, VGG, Inception, and ResNet \cite{krizhevsky2012imagenet, simonyan2014very, szegedy2015going, he2016deep}.
While these architectures steadily improved top-1 accuracy on ImageNet \cite{deng2009imagenet},
their memory footprint and computational cost grew proportionally,
prompting a parallel line of research into efficiency-focused approaches.
Lightweight backbones such as MobileNet \cite{howard2017mobilenets,sandler2018mobilenetv2},
EfficientNet \cite{tan2019efficientnet}, and the Data-efficient Vision Transformer (DeiT) \cite{touvron2021training}
demonstrated that accuracy can be retained through careful depth-width scaling and attention re-use.
Nevertheless, these models still require substantial training resources when trained from scratch.

\subsection{Knowledge Distillation}
\label{sec:rw:knowledge_distillation}

Knowledge distillation is a technique used to transfer knowledge
from a large, complex model to a smaller, more efficient model,
enabling deployment on resource-constrained devices without significant performance loss.
Among several aspects of knowledge distillation,
the paper specifically focuses on the mimicry between the teacher and student models,
which enables the small model to follow the teacher's probability distribution.
Hinton et al. \cite{hinton2015distilling} introduced logit-matching between a cumbersome teacher and a compact student.
Subsequent works enriched the transfer signal by aligning intermediate feature maps \cite{romero2014fitnets}
or the spatial attention of convolutional layers \cite{zagoruyko2016paying}.
Another notable approach, self-distillation \cite{zhang2019your},
has been shown to be effective in improving the performance of a single model
by training it with its own architecture.

\subsection{Quantization}
\label{sec:rw:quantization}

Quantization techniques have recently gained significant attention for their ability
to enable \gls*{llm} deployment on low-compute devices.
In particular, Microsoft\footnote{https://www.microsoft.com/en-us/} has pioneered the quantization of \gls*{llm}
through their Phi-3 \cite{microsoft_phi3} and Phi-4 \cite{microsoft_phi4} models,
as well as the 1-bit quantized model, BitNet \cite{microsoft_bitnet}.
Floating-point quantization techniques can also be applied to the image classification task,
enabling the deployment of large models on low-compute devices.
