\documentclass[conference]{IEEEtran}
\IEEEoverridecommandlockouts
% The preceding line is only needed to identify funding in the first footnote. If that is unneeded, please comment it out.
%Template version as of 6/27/2024

\bibliographystyle{IEEEtran}

\usepackage{cite}
\usepackage{amsmath,amssymb,amsfonts}
\usepackage{algorithmic}
\usepackage{graphicx}
\usepackage{textcomp}
\usepackage{xcolor}
\def\BibTeX{{\rm B\kern-.05em{\sc i\kern-.025em b}\kern-.08em
    T\kern-.1667em\lower.7ex\hbox{E}\kern-.125emX}}

\usepackage{hyperref}
\usepackage[acronym]{glossaries}

\makeglossaries

\newacronym{gan}{GAN}{Generative Adversarial Network}
\newacronym{cnn}{CNN}{Convolutional Neural Network}
\newacronym{llm}{LLM}{Large Language Model}

\newcommand{\WP}[1]{\textcolor{red}{(Wonjun: #1)}}

\begin{document}

\title{
    Knowledge Distilation for Smaller Models\\in Computer Vision
}

\author{
\IEEEauthorblockN{\textbf{Wonjun Park\textsuperscript{*}, Abhinay Kotla\textsuperscript{*}}}\thanks{* Equal Contribution.}
\thanks{This paper is a part of the final project of the course for Spring 2025, CSE-5368-001: Neural Networks at the University of Texas at Arlington under Dr. Francklin Rivas.}
\IEEEauthorblockA{Computer Science and Engineering \\
University of Texas at Arlington \\
Arlington, TX, USA \\
\textit{ \{ wxp7177, axk5827 \}@mavs.uta.edu} }
}

\maketitle

\begin{abstract}
    % Diffusion Models have been successfully shown their capabilities to restore images from masked regions.
    % Nevertheless, due to the scale of their sizes and parameters,
    % leveraging those models in low computational devices like smartphones is still a challenge
    % despite its advantages in terms of privacy and security.
    % In this paper, TinyDiff, a quantized and distilled version of the diffusion model, is proposed to address this issue.
    % The model is significantly small in terms of the number of parameters and the size, preserving the competitive performance.
    % With this model, we expect that users with low computational devices are able to take advantages of both the diffusion model and the edge devices.
    % The code is available at \href{https://github.com}{here}.
    Knowledge distilation is a technique that allows deep learning models
    to transfer knowledge from a larger, and more complex model (the teacher)
    to a smaller, and more efficient model (the student).
    The technique is able to be adopted in various tasks
    including image classification and image generation.
    Among the tasks in computer vision,
    the paper specifically focuses on the image classification task
    to demonstrate the effectiveness of knowledge distilation,
    further outlining the framework to apply the technique to other tasks.
    The results from the image classification task that addressed in this paper
    shows that the knowledge distilation is able to successfully decrease the size of the model
    while preserving the performance of the model.
    Notably, the distilled model achieves slightly better performance than the teacher model.
    We expect that the framework can be applied to other tasks in computer vision without performance degradation.
\end{abstract}

\begin{IEEEkeywords}
    deep learning, computer vision, knowledge distillation, teacher-student learning, and quantization.
\end{IEEEkeywords}

\section{Introduction}

\cite{nazeri2019edgeconnect}

\section{Related Works}

In this section, we review performance optimization techniques
for the image classification task through three main approaches:
\textit{\ref{sec:rw:lightweight_backbones} lightweight backbones},
\textit{\ref{sec:rw:knowledge_distillation} knowledge distillation},
and \textit{\ref{sec:rw:quantization} quantization}.
The first approach focuses on the design of efficient architectures,
the second emphasizes the transfer of knowledge from a larger model to a smaller one,
and the third discusses the quantization of models,
a widely used technique for reducing model size
and improving performance on low-compute devices.

\subsection{Lightweight Backbones}
\label{sec:rw:lightweight_backbones}

Early breakthroughs in large-scale image classification were driven by deeper and wider \gls*{cnn}
architectures such as AlexNet, VGG, Inception, and ResNet \cite{krizhevsky2012imagenet, simonyan2014very, szegedy2015going, he2016deep}.
While these architectures steadily improved top-1 accuracy on ImageNet \cite{deng2009imagenet},
their memory footprint and computational cost grew proportionally,
prompting a parallel line of research into efficiency-focused approaches.
Lightweight backbones such as MobileNet \cite{howard2017mobilenets,sandler2018mobilenetv2},
EfficientNet \cite{tan2019efficientnet}, and the Data-efficient Vision Transformer (DeiT) \cite{touvron2021training}
demonstrated that accuracy can be retained through careful depth-width scaling and attention re-use.
Nevertheless, these models still require substantial training resources when trained from scratch.

\subsection{Knowledge Distillation}
\label{sec:rw:knowledge_distillation}

Knowledge distillation is a technique used to transfer knowledge
from a large, complex model to a smaller, more efficient model,
enabling deployment on resource-constrained devices without significant performance loss.
Among several aspects of knowledge distillation,
the paper specifically focuses on the mimicry between the teacher and student models,
which enables the small model to follow the teacher's probability distribution.
Hinton et al. \cite{hinton2015distilling} introduced logit-matching between a cumbersome teacher and a compact student.
Subsequent works enriched the transfer signal by aligning intermediate feature maps \cite{romero2014fitnets}
or the spatial attention of convolutional layers \cite{zagoruyko2016paying}.
Another notable approach, self-distillation \cite{zhang2019your},
has been shown to be effective in improving the performance of a single model
by training it with its own architecture.

\subsection{Quantization}
\label{sec:rw:quantization}

Quantization techniques have recently gained significant attention for their ability
to enable \gls*{llm} deployment on low-compute devices.
In particular, Microsoft\footnote{https://www.microsoft.com/en-us/} has pioneered the quantization of \gls*{llm}
through their Phi-3 \cite{microsoft_phi3} and Phi-4 \cite{microsoft_phi4} models,
as well as the 1-bit quantized model, BitNet \cite{microsoft_bitnet}.
Floating-point quantization techniques can also be applied to the image classification task,
enabling the deployment of large models on low-compute devices.


\section{Methodology}

\cite{xia2023diffir}

\subsection{Knowledge Distillation}



\subsection{Quantization}

\section{Experiments}

\subsection{Training Details}

\cite{zhou2017places}

\subsection{Evaluation}

\gls{fid} \cite{heusel2017gans}

\section{Conclusion}

\bibliography{final-paper}

\printglossary[type=\acronymtype]

\end{document}
