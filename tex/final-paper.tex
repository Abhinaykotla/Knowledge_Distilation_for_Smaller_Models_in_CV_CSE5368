\documentclass[conference]{IEEEtran}
\IEEEoverridecommandlockouts
% The preceding line is only needed to identify funding in the first footnote. If that is unneeded, please comment it out.
%Template version as of 6/27/2024

\bibliographystyle{IEEEtran}

\usepackage{cite}
\usepackage{amsmath,amssymb,amsfonts}
\usepackage{algorithmic}
\usepackage{graphicx}
\usepackage{textcomp}
\usepackage{xcolor}
\def\BibTeX{{\rm B\kern-.05em{\sc i\kern-.025em b}\kern-.08em
    T\kern-.1667em\lower.7ex\hbox{E}\kern-.125emX}}

\usepackage{hyperref}
\usepackage[acronym]{glossaries}

\makeglossaries

\newacronym{gan}{GAN}{Generative Adversarial Network}
\newacronym{cnn}{CNN}{Convolutional Neural Network}
\newacronym{llm}{LLM}{Large Language Model}

\newcommand{\WP}[1]{\textcolor{red}{(Wonjun: #1)}}

\begin{document}

\title{
    Knowledge Distilation for Smaller Models\\in Computer Vision
}

\author{
\IEEEauthorblockN{\textbf{Wonjun Park\textsuperscript{*}, Abhinay Kotla\textsuperscript{*}}}\thanks{* Equal Contribution.}
\thanks{This paper is a part of the final project of the course for Spring 2025, CSE-5368-001: Neural Networks at the University of Texas at Arlington under Dr. Francklin Rivas.}
\IEEEauthorblockA{Computer Science and Engineering \\
University of Texas at Arlington \\
Arlington, TX, USA \\
\textit{ \{ wxp7177, axk5827 \}@mavs.uta.edu} }
}

\maketitle

\begin{abstract}
    % Diffusion Models have been successfully shown their capabilities to restore images from masked regions.
    % Nevertheless, due to the scale of their sizes and parameters,
    % leveraging those models in low computational devices like smartphones is still a challenge
    % despite its advantages in terms of privacy and security.
    % In this paper, TinyDiff, a quantized and distilled version of the diffusion model, is proposed to address this issue.
    % The model is significantly small in terms of the number of parameters and the size, preserving the competitive performance.
    % With this model, we expect that users with low computational devices are able to take advantages of both the diffusion model and the edge devices.
    % The code is available at \href{https://github.com}{here}.
    Knowledge distilation is a technique that allows deep learning models to transfer knowledge from a larger, more complex model (the teacher)
    to a smaller, more efficient model (the student).
    The technique can be adopted in various tasks including image classification and image generation.
    Among computer vision tasks, the paper specifically focuses on the image classification task to demonstrate the effectiveness of knowledge distillation, further highlighting the framework to apply the technique to other tasks.
    The results from the image classification task that is addressed in this paper
show that knowledge distillation is able to successfully decrease the size of the model while preserving the performance of the model.
    In particular, the distilled model achieves slightly better performance than the teacher model.
    We expect that the framework can be applied to other tasks in computer vision without affecting performance.
    The code is available at \href{https://github.com/Abhinaykotla/Knowledge_Distilation_for_Smaller_Models_in_CV_CSE5368}{here}.
\end{abstract}

\begin{IEEEkeywords}
    deep learning, computer vision, knowledge distillation, teacher-student learning, and quantization.
\end{IEEEkeywords}

\section{Introduction}

Deep neural networks have achieved remarkable success across a wide spectrum of computer vision problems,
from a simple image classification task, ImageNet \cite{deng2009imagenet} for example, to high-fidelity image generation with diffusion \cite{ho2020denoising, rombach2022high, xia2023diffir} and \gls*{gan} models \cite{goodfellow2014generative, nazeri2019edgeconnect}.
Much of this progress, however, has been powered by ever-larger model capacities;
billions of parameters, expansive training datasets, and considerable computational budgets,
even though such computationally expensive models deliver state-of-the-art performance.
Their computational-intensive nature poses serious obstacles for deploying them
on resource-constrained devices and real-time and latency-sensitive backend services
such as smartphones or even low-end GPUs.
Closing this gap between accuracy and efficiency therefore becomes one of central research questions in deep learning.

Image classification is a basic and fundamental task in computer vision.
It is the task of predicting the class of an image from a set of classes.
Many other tasks like object detection, image segmentation, and image generation
are successfully built on top of image classification
since AlexNet \cite{krizhevsky2012imagenet} overwhelmed the traditional methods
in computer vision with its deep learning-based approach.

In this project, the initial plan was to implement a knowledge distillation framework
for an image inpainting task with diffusion models
which restores images from masked regions.
Unfortunately, however, while training scripts for student models were successfully implemented and run with the guidance of the teacher model from DiffIR \cite{xia2023diffir},
not only the problem of the memory but also the problem of the computing operations
led the training time to be extremely long.
More than 200 hours were required per epoch for training on the Places dataset \cite{zhou2017places} with our student model from DiffIR,
even if it operated on 8-bit floating point.
Therefore, we decided to change the task from image inpainting to image classification
to demonstrate the effectiveness of knowledge distillation
and to further the foundation of the framework to apply the technique in the future.

Based on our code implemented in Assignment 2 of the course, CSE 5368,
we built a knowledge distillation framework for the image classification task.
In the subsequent sections, the paper describes our knowledge distillation method,
presents experimental results demonstrating its efficiency,
and discusses potential directions for future work.

\section{Related Works}

\subsection{Image Inpainting} 

\subsection{Diffusion Models}

% \subsection{Knowledge Distillation}

% \subsection{Quantization}

\section{Methodology}

\subsection{Intel Image Dataset}

The Intel Image Classification Dataset \cite{intel_image_classification_kaggle} was curated
as part of a data hackathon organized
by Intel \footnote{https://www.intel.com/content/www/us/en/homepage.html}
on an online platform Kaggle \footnote{https://www.kaggle.com/},
with the aim of fostering engagement and innovation within data science communities.
The dataset comprises 25,000 RGB color images,
each with a resolution of 150 $\times$ 150 pixels,
depicting a variety of natural and urban scenes from around the world.
The images, originally captured by Jan Böttinger
on Unsplash \footnote{https://unsplash.com/},
are labeled into six distinct classes: buildings, forest, glacier, mountain, sea, and street.

The dataset is partitioned into three subsets: approximately 14,000 images for training,
3,000 images for testing, and 7,000 images for prediction tasks.
Each subset is provided as a separate zipped file through Kaggle.

The primary object of this dataset is to support participants
in developing and evaluating image classification models.
By providing a diverse and well-labeled collection of scenes,
the dataset encourages the advancement of robust classification algorithms
capable to distinguish objects in various environments with reasonable accuracy.

\subsection{Model Architecture}
\label{sec:method:model_architecture}

In this section, the architecture of the teacher model is described.
We implemented a custom \gls*{cnn} model, including the residual block \cite{he2016deep},
which hierarchically extracts and aggregates spatial features from the input images,
before mapping them to one of six scene categories.

The model consists of 12 residual blocks,
progressively increasing the channel capacity while reducing spatial resolution,
and a fully connected layer at the end as a classifier
which regularly decreases the number from 1024, 512, 256, 64, and 6.
In total, 26,534,358 parameters are in the model.

All operations in the model are performed using floating point 32,
which is the default precision in PyTorch \cite{paszke2019pytorch}
in most GPUs.

\subsection{Weight Reduction}
\label{sec:method:weight_reduction}

From the teacher model,
the number of residual blocks and the layers of the classifier are reduced.
In the perspective of weight reduction,
four reduction models are created, excluding the teacher model,
which are 10, 8, 6, and 4 residual blocks
and (512, 256, 64, 16), (256, 64, 16), (128, 64, 16), and (64, 32) layers of the classifiers.
The number of parameters in the models is 6,601,686, 1,648,470, 408,854, and 98,294, respectively.

\subsection{Precision Quantization}
\label{sec:method:precision_quantization}

The quantization is utilized in both model weights and floating point operations.
Two different levels of precision, floating point 16 and 8, are used.
To implement the quantization,
\texttt{torch.set\_default\_dtype()} and the \texttt{bnb}~\cite{bitsandbytes} Python framework
are used.

\section{Experiments}

\subsection{Training Details}

Several student models were configured based on the teacher model described in Section~\ref{sec:method:model_architecture},
applying the reduction and quantization techniques
introduced in Sections~\ref{sec:method:weight_reduction} and~\ref{sec:method:precision_quantization}.
All training was conducted on a single NVIDIA GeForce RTX 4070 Laptop GPU.
The implementation is based on the PyTorch framework \cite{paszke2019pytorch},
leveraging its extensive libraries and tools for deep learning.
The Adam optimizer \cite{kingma2014adam} was employed with a learning rate of 0.001.
A mini-batch size of 224 was used, and KL Divergence was utilized as the loss function,
effectively measuring the difference
between the distributions predicted by the teacher and those output by the student models.
Specifically, the the loss quantifies how well the student model mimics the teacher model's output distribution,
minimized when the student model closely approximates the teacher model's output distribution.
The following formulations shows the KL Divergence,

\begin{equation}
    \mathrm{KL}\bigl(P\;\|\;Q\bigr)
    = \sum_{c=1}^{C} P(c\!\mid\!x)\,
        \log\frac{P(c\!\mid\!x)}
                 {Q(c\!\mid\!x)}\,,
\label{eq:kl_divergence}
\end{equation}

where $P$ is the teacher's distribution and $Q$ is the student's distribution. Early stopping was applied with a patience of 3 epochs to prevent overfitting.


\subsection{Evaluation}

The performance of the teacher model is summarized in Table~\ref{tab:teacher_model}.
Table~\ref{tab:weight_reduction} shows the performance of the student models
along with the reduction in the number of parameters.
Additionally, Table~\ref{tab:precision_quantization} presents the performance
achieved through quantization of model weights and floating-point operations.

\begin{table}[ht]
\centering
\caption{Teacher Model Performance}
\label{tab:teacher_model}
\begin{tabular}{c|c}
    \noalign{\hrule height 1pt}
                            & \textit{Teacher model} \\ \hline
    Number of parameters    & 26,534,358 \\ 
    Accuracy                 & 85.77\% \\ 
    Training time per epoch (seconds) & 72.32 \\ 
    \noalign{\hrule height 1pt}
\end{tabular}
\end{table}

\begin{table}[ht]
\centering
\caption{Weight Reduction Comparison}
\label{tab:weight_reduction}
\begin{tabular}{c|c}
    \noalign{\hrule height 1pt}
                            & \textit{10 blocks, (512, 256, 64, 16)} \\ \hline
    Number of parameters    & 6,601,686 \\ 
    Accuracy                 & 83.23\% \\ 
    Training time per epoch (seconds) & 68.6 \\ 
    \noalign{\hrule height 1pt}
\end{tabular}
\\[10pt]
\begin{tabular}{c|c}
    \noalign{\hrule height 1pt}
                            & \textit{8 blocks, (256, 64, 16)} \\ \hline
    Number of parameters    & 1,648,470 \\ 
    Accuracy                 & \textbf{86.43\%} \\ 
    Training time per epoch (seconds) & 67.0 \\ 
    \noalign{\hrule height 1pt}
\end{tabular}
\\[10pt]
\begin{tabular}{c|c}
    \noalign{\hrule height 1pt}
                            & \textit{6 blocks, (128, 64, 16)} \\ \hline
    Number of parameters    & 408,854 \\ 
    Accuracy                 & 85.77\% \\ 
    Training time per epoch (seconds) & 66.7 \\ 
    \noalign{\hrule height 1pt}
\end{tabular}
\\[10pt]
\begin{tabular}{c|c}
    \noalign{\hrule height 1pt}
                            & \textit{4 blocks, (64, 32)} \\ \hline
    Number of parameters    & 98,294 \\ 
    Accuracy                 & 84.57\% \\ 
    Training time per epoch (seconds) & \textbf{64.7} \\ 
    \noalign{\hrule height 1pt}
\end{tabular}
\end{table}

\begin{table}[ht]
\centering
\caption{Precision Quantization on Teacher Model}
\label{tab:precision_quantization}
\begin{tabular}{c|cc}
    \noalign{\hrule height 1pt}
                        & \textit{fp16}        & \textit{fp8} \\ \hline
    Accuracy            & 85.54\%              & \textbf{86.38\%} \\ 
    Training time per epoch (seconds) & \textbf{61.3}    & 70.5 \\ 
    \noalign{\hrule height 1pt}
\end{tabular}
\end{table}

As expected, the training time per epoch decreases as the number of parameters is reduced.
The smallest student model achieves the fastest training time per epoch, at 64.7 seconds.
Interestingly, the performance does not strictly correlate with the number of parameters:
the 8-block student model outperforms others with an accuracy of 86.43\%,
highlighting that architectural choices and dataset characteristics
can outweigh mere parameter counts.

\begin{table}[ht]
\centering
\caption{Precision Quantization on 8-Block Model}
\label{tab:quantization_8_blocks}
\begin{tabular}{c|cc}
    \noalign{\hrule height 1pt}
                        & \textit{fp16}        & \textit{fp8} \\ \hline
    Accuracy            & \textbf{87.00\%}     & 86.07\% \\ 
    Training time per epoch (seconds) & \textbf{60.8}    & 69.0 \\ 
    \noalign{\hrule height 1pt}
\end{tabular}
\end{table}

Using the best-performing student model (8 blocks),
we further experimented with precision quantization.
The results in Table~\ref{tab:quantization_8_blocks} demonstrate that
the 8-block student model benefits significantly from fp16 precision.
It achieved an accuracy of 87.00\% with the fastest training time per epoch at 60.8 seconds,
while fp8 precision resulted in slightly lower accuracy (86.07\%) and a slower training time (69.0 seconds).
This highlights that selecting the appropriate precision can significantly improve
both model performance and computational efficiency.


\section{Conclusion}

In this paper, we have presented a comprehensive study
on the parameter reduction and quantization of image classification models using knowledge distillation.
We have demonstrated that knowledge distillation is an effective technique
for transferring knowledge from a larger, more complex model (the teacher)
to a smaller, more efficient model (the student)
in the context of image classification tasks.
Based on this result, we believe that our knowledge distilation framework can also be applied
to other tasks in computer vision including image inpainting
without significant performance degradation.
We expect that this framework will pave the way for the development of other efficient models
in various computer vision tasks.

\bibliography{final-paper}

\printglossary[type=\acronymtype]

\end{document}
